\documentclass{article}
\usepackage{amsmath}
\usepackage{amssymb}
\usepackage{amsthm}
\begin{document}
\section{Finite automata}
A \textbf{finite automaton} is a 5-tuple $(Q, \Sigma, \delta, q_0, F)$ where:
\begin{itemize}
	\item $Q$ is a finite set of states
	\item $\Sigma$ is the finite alphabet
	\item $\delta: Q\times\Sigma \rightarrow Q$ is the transition function
	\item $q_0 \in Q$ is the start state
	\item $F \subseteq Q$ is the set of accept states (final states)
\end{itemize}
A finite automaton reads an input cell-by-cell, transitioning between states depending on the current input symbol and state.\medskip
\\ If $A$ is the set of all strings accepted by a machine $M$, $A$ is the \textbf{language} of $M$, denoted $L(M) = A$. If $A$ is the language of $M$, $M$ \textbf{recognises} $A$.\medskip
\\Can also say that $M$ accepts $A$, but this is confusing -- say "accept" when referring to machines accepting strings, and "recognise" when referring to machines accepting languages.\medskip
\\ A machine only recognises one language, but can accept several strings. If a machine accepts no strings, it recognises the empty language $\emptyset$.\medskip
\\\textbf{Question.} Is zero accept states allowable?
\\\textbf{Answer.} Yes. Set $F$ to be the empty set $\emptyset$, yielding zero accept states.\medskip
\\\textbf{Question.} Must there be exactly one transition exiting every state for each possible input symbol?
\\\textbf{Answer.} Yes. The transition function, $\delta$, specifies one successor state for each possible combination of a state and an input symbol.\medskip
\\Finite automata are good models for computers with extremely limited memory.

\subsection{Computation}
Let $M = (Q, \Sigma, \delta, q_0, F)$ be a finite automaton and $w = w_1w_2...w_n$ be a string, where each $w_i$ is a member of alphabet $\Sigma$. $M$ accepts $w$ is there exists a sequence of states $r_0, r_1, ..., r_n \in Q$ where:
\begin{enumerate}
	\item $r_0 = q_0$ (machine starts in the start state),
	\item $\delta(r_i, w_{i+1}) = r_{i+1}$ for $i = 0, ..., n-1$ (machine goes between states according to the transition function), and
	\item $r_n \in F$ (machine accepts an input if it ends in an accept state).
\end{enumerate}
The machine $M$ recognises language $A$ if $A = \{w|M\text{ accepts } w\}$

\section{Nondeterminism}
In deterministic computation, when a machine is in a given state and reads the next input symbol, its next state is known (determined). In nondeterministic machines, there can be several choices for the next state.\medskip
\\ Nondeterminism is just a generalisation of determinism. Every deterministic finite automaton (DFA) is automatically a nondeterministic finite automaton (NDFA).\medskip
\\ Every state of a DFA has one exiting transition arrow for each symbol in the alphabet. In NDFAs, a state can have zero, one, or many exiting arrows for each symbol in the alphabet.\medskip
\\ In DFAs, labels on transition arrows must be symbols from the alphabet. In NDFAs, labels on transition arrows may be symbols from the alphabet, or $\epsilon$. There can be zero, one, or many arrows exiting each state with label $\epsilon$.

\subsection{How do nondeterministic machines compute?}
Suppose a NDFA is running on an input string and gets to a state with multiple ways to proceed. After reading the input symbol, the machine splits into multiple copies of itself, and follows all possibilities in parallel. Each new machine takes one of the possibilities and continues.\medskip
\\ If $\epsilon$-labelled exiting arrows are encountered, the machine splits into multiple copies -- one for each of the $\epsilon$-labelled arrows and one staying at the current state -- without reading an input.\medskip
\\ In any of the new machines, if the next input symbol does not appear on any arrows exiting the current state, the machine (and the branch of computation associated with it) dies.\medskip
\\ If any one of these machines is in an accept state at the end of the input, the input string is accepted by the NDFA.\medskip
\\\textbf{Summary.} Nondeterministic computation can viewed as a tree of possibilities, with the root being the start of the computation, and every branching point (fork) corresponds to a point where the machine has multiple choices.\medskip
\\ If one of the branches ends in an accept state, the machine accepts.\medskip
\\Nondeterminism can be considered parallel computation wherein multiple independent "processes" or "threads" can be running concurrently.

\subsection{Nondeterministic finite automata}
DFAs and NDFAs differ in the type of transition function:
\begin{itemize}
	\item In DFAs, the transition function takes a state and an input symbol, and produces the next state
	\item In NDFAs, the transition function takes a state and an input symbol or the empty string, and produces the set of possible next states
\end{itemize}
A \textbf{nondeterministic finite automaton} is a 5-tuple $(Q, \Sigma, \delta, q_0, F)$ where:
\begin{itemize}
	\item $Q$ is a finite set of states
	\item $\Sigma$ is the finite alphabet
	\item $\delta: Q\times\Sigma_{\epsilon} \rightarrow \mathcal{P}(Q)$ is the transition function
	\item $q_0 \in Q$ is the start state
	\item $F \subseteq Q$ is the set of accept states (final states)
\end{itemize}
$\mathcal{P}(Q)$ is the power set (set of all subsets) of a set $Q$. For any alphabet $\Sigma$, write $\Sigma_{\epsilon}$ to mean $\Sigma \cup \{\epsilon\}$.\medskip
\\The \textbf{formal definition of computation} for a NDFA is also similar for that of a DFA. Let $N = (Q, \Sigma, \delta, q_0, F)$ be a NDFA and $w$ be a string over alphabet $\Sigma$. $N$ accepts $w$ if $w$ can be written as $w = y_1y_2...y_m$, where each $y_i$ is a member of $\Sigma_{\epsilon}$ and there exists a sequence of states $r_0, r_1, ..., r_m \in Q$ where:
\begin{enumerate}
	\item $r_0 = q_0$ (machine starts in the start state),
	\item $r_{i+1} \in \delta(r_i, y_{i+1})$, for $i = 0, ..., m-1$ (state $r_{i+1}$ is one of the allowable next states when $N$ is in state $r_i$ and reading $y_{i+1}$), and
	\item $r_m \in F$ (machine accepts an input if it ends in an accept state).
\end{enumerate}
Note that $\delta(r_i, y_{i+1})$ is the set of allowable next states, so say that $r_{i+1}$ is a member of this set.

\subsection{How are NDFAs useful?}
\begin{itemize}
	\item Every NDFA can be converted into an equivalent DFA -- it is sometimes easier to construct NDFAs than directly constructing DFAs
	\item NDFAs can be much smaller than their DFA counterparts
	\item The functioning of some NDFAs is easier to understand than that of their DFA counterparts
	\item "Nondeterminism in finite automata is a good introduction to nondeterminism in more powerful computational models"
\end{itemize}

\subsection{Equivalence}
Two machines are \textbf{equivalent} if they recognise the same language.\medskip
\\\textbf{Theorem.} Every nondeterministic finite automaton has an equivalent deterministic finite automaton.\medskip
\\ If $k$ is the number of states of the NDFA, it has $2^k$ subsets of states. Each subset corresponds to one of the possibilities that the DFA must remember, so there will be $2^k$ states in the DFA simulating the NDFA.\medskip
\\\textbf{Proof.} Let $N = (Q, \Sigma, \delta, q_0, F)$ be the NDFA recognising some language $A$. Construct DFA $M = (Q', \Sigma, \delta', q_0', F')$ recognising $A$.\medskip
\\For any state $R$ of $M$, define $E(R)$ to be the collection of states that can be reached from members of $R$ only by going along $\epsilon$ arrows, including the members of $R$ themselves. For $R \subseteq Q$ let:
$$E(R) = \{q | q \text{ can be reached from } R \text{ by travelling along 0 or more } \epsilon \text{ arrows}\}$$
The construction of $M$ is as follows:
\begin{itemize}
	\item $Q' = \mathcal{P}(Q)$ -- every state of $M$ is a set of states in $N$.
	\item For $R \in Q'$ and $a \in \Sigma$, let $\delta'(R, a) = \{q\in Q | q \in E(\delta(r, a)) \text{ for some } r \in R\}$ -- if $R$ is a state of $M$, it is also a set of states in $N$. When $M$ reads a symbol $a$ in state $R$, it shows where $a$ takes each state in $R$. As each state may go to a set of states, take the union of all these sets.
	\item $q_0' = E(\{q_0\})$ -- M starts in the state corresponding to the set containing just the start state of $N$.
	\item $F' = \{R \in Q' | R\text{ contains an accept state of } N\}$ -- $M$ accepts if one of the possible states that $N$ could be in at any point is an accept state.\hfill\qedsymbol
\end{itemize}

\section{Regular languages}
A language is a \textbf{regular language} if it is recognised by some finite automaton.\medskip
\\\textbf{Theorem.} A language is regular if and only if some nondeterministic finite automaton recognises it.\medskip
\\\textbf{Proof.}
\begin{align*}
(\rightarrow)\qquad&\qquad\parbox[t]{0.8\textwidth}{As any NDFA can be converted into an equivalent DFA, if an NDFA recognises some languages, so does some DFA, and the language is regular.} \\
(\leftarrow)\qquad&\qquad\parbox[t]{0.8\textwidth}{A regular language has a DFA recognising it, and any DFA is also an NDFA.}
\end{align*}

\subsection{Regular operations}
Define three operations and language, called \textbf{regular operations}, to study properties of the regular languages.\medskip
\\ Let $A$ and $B$ be languages.
\begin{itemize}
	\item \textbf{Union:} $A \cup B = \{x | x \in A\text{ or } x \in B\}$
	\\ Takes all strings in both $A$ and $B$ and puts them in one language.
	\item \textbf{Concatenation:} $A \circ B = \{xy | x \in A\text{ and } y \in B\}$
	\\ Attaches a string from $A$ in front of a string from $B$, in all possible ways, to get the strings in the new language.
	\item \textbf{Star:} $A^* = \{x_1x_2...x_k | k \geq 0\text{ and each } x_i \in A\}$
	\\ A unary operation that attaches any number of strings in $A$ together to get a string in the new language.
	\\ As "any number" includes zero, $\epsilon$ is always a member of $A^*$.
\end{itemize}

\subsection{Closure under operations}
\textbf{Theorem.} The class of regular languages is closed under union.\medskip
\\\textbf{Proof.} Let $N_1 = (Q_1, \Sigma, \delta_1, q_1, F_1)$ recognise $A_1$ and $N_2 = (Q_2, \Sigma, \delta_2, q_2, F_2)$ recognise $A_2$. Construct $N = (Q, \Sigma, \delta, q_0, F)$ to recognise $A_1 \cup A_2$.
\begin{enumerate}
	\item $Q = \{q_0\} \cup Q_1 \cup Q_2$
	\\ The states of $N$ are all the states of $N_1$ and $N_2$, with the addition of a new start state, $q_0$.
	\item State $q_0$ is the start state of N
	\item The set of accept states is $F = F_1 \cup F_2$
	\\ N accepts if either $N_1$ or $N_2$ accepts, so the accept states of $N$ are all those of $N_1$ and $N_2$.
	\item Define $\delta$ so that for any $q \in Q$ and any $a \in \Sigma_{\epsilon}$:
	\begin{equation*}
	\delta(q, a) = \begin{cases}
		\delta_1(q, a) & q \in Q_1\\
		\delta_2(q, a) & q \in Q_2\\
		\{q_1, q_2\} & q = q_0 \text{ and } a = \epsilon\\
		\emptyset & q = q_0 \text{ and } a \neq \epsilon
	\end{cases}
	\end{equation*}
\end{enumerate}
\textbf{Theorem.} The class of regular languages is closed under concatenation.\medskip
\\\textbf{Proof.} Let $N_1 = (Q_1, \Sigma, \delta_1, q_1, F_1)$ recognise $A_1$ and $N_2 = (Q_2, \Sigma, \delta_2, q_2, F_2)$ recognise $A_2$. Construct $N = (Q, \Sigma, \delta, q_1, F_2)$ to recognise $A_1 \circ A_2$.
\begin{enumerate}
	\item $Q = Q_1 \cup Q_2$
	\item The state $q_1$ is the same as the start state of $N_1$
	\item Accept states $F_2$ are the same as the accept states of $N_2$.
	\\ The accept states of $N_1$ have additional $\epsilon$ arrows that nondeterministically allow branching to $N_2$ whenever $N_1$ is in an accept state, signifying that $N_1$ has found an initial piece of the input that constitutes a string in $A_1$.
	\\ $N$ accepts when the input can be split into two parts -- the first accepted by $N_1$ and the second by $N_2$.
	\\ $N$ nondeterministically "guesses" where to make the split.
	\item Define $\delta$ so that for any $q \in Q$ and any $a \in \Sigma_{\epsilon}$:
	\begin{equation*}
		\delta(q, a) = \begin{cases}
		\delta_1(q, a) & q \in Q_1 \text{ and } q \notin F_1\\
		\delta_1(q, a) & q \in F_1 \text{ and } a \neq \epsilon\\
		\delta_1(q, a) \cup \{q_2\} & q \in F_1 \text{ and } a = \epsilon\\
		\delta_2(q, a) & q \in Q_2 
		\end{cases}
	\end{equation*}
\end{enumerate}
\textbf{Theorem.} The class of regular languages is closed under star.\medskip
\\ Given a regular language $A_1$, want to prove that $A_1^*$ is also regular. Take a NDFA $N_1$ recognising $A_1$ and modify it to recognise $A_1^*$. The resulting NDFA $N$ will accept its input whenever it can be broken into several pieces, with $N_1$ accepting each piece.\medskip
\\ Construct $N$ like $N_1$ with additional $\epsilon$ arrows returning to the start state from the accept states.
\\ When processing gets to the end of a piece that $N_1$ accepts, $N$ has the option of jumping back to the start state to try and read another piece that $N_1$ accepts.
\\ $N$ must also accept $\epsilon$ as this is always a member of $A_1^*$.\medskip
\\ Could add the start state to the set of accept states, but this may also add undesired strings to the recognised language, rather than just $\epsilon$. Instead, add a new start state (which is also an accept state), and has an $\epsilon$ arrow to the old start state. This adds $\epsilon$ to the language without adding anything else.\medskip
\\\textbf{Proof.} Let $N_1 = (Q_1, \Sigma, \delta_1, q_1, F_1)$ recognise $A_1$. Construct $N = (Q, \Sigma, \delta, q_0, F)$ to recognise $A_1^*$.
\begin{enumerate}
	\item $Q = Q_1 \cup \{q_0\}$
	\\ The states of $N$ are those of $N_1$, plus the new start state
	\item $q_0$ is the new start state
	\item $F = F_1 \cup \{q_0\}$
	\\ The accept states are the old accept states, plus the new start state
	\item Define $\delta$ so that for any $q \in Q$ and any $A \in \Sigma_{\epsilon}$:
	\begin{equation*}
		\delta(q, a) = \begin{cases}
			\delta_1(q, a) & q \in Q_1 \text{ and } q \notin F_1\\
			\delta_1(q, a) & q \in F_1 \text{ and } a \neq \epsilon\\
			\delta_1(q, a) \cup \{q_1\} & q \in F_1 \text{ and } a = \epsilon\\
			\{q_1\} & q = q_0 \text{ and } a = \epsilon\\
			\emptyset & q = q_0 \text{ and } a \neq \epsilon
		\end{cases}
	\end{equation*}
\end{enumerate}

\section{Regular expressions}
Use regular operations to build up expressions describing languages, which are called \textbf{regular expressions}.\medskip
\\ Say that $R$ is \textbf{regular expression} if $R$ is:
\begin{enumerate}
	\item $a$ for some $a$ in alphabet $\Sigma$,
	\\ Regular expression $a$ represents language $\{a\}$
	\item $\epsilon$,
	\\ Regular expression $\epsilon$ represents language $\{\epsilon\}$
	\item $\emptyset$,
	\\ Represents the empty language
	\item $(R_1 \cup R_2)$, where $R_1$ and $R_2$ are regular expressions,
	\item $(R_1 \circ R_2)$, where $R_1$ and $R_2$ are regular expressions, or
	\item $(R_1^*)$, where $R_1$ is a regular expression.
\end{enumerate}
Note that $\epsilon$ represents the language containing a single string--the empty string--whereas $\emptyset$ represents the language that does not contain any strings.\medskip
\\ For convenience, let $R^+$ be shorthand for $RR^*$. $R^*$ has all strings that are 0 or more concatenations of strings from $R$. $R^+$ has all strings that are 1 or more concatenations of strings from $R$. $R^+ \cup \epsilon = R^*$.\medskip
\\ Let $R^k$ be shorthand for the concatenation of $k$ $R$'s with each other.\medskip
\\ As $R_1$ and $R_2$ are always smaller than $R$, a circular definition is avoided (defining the notion of a regular expression in terms of itself). Defining regular expressions in terms of smaller regular expressions avoids circularity -- this is an inductive definition.\medskip
\\ To distinguish between a regular expression $R$ and the language it describes, use $L(R)$ to describe the language of $R$.\medskip
\\ The order of operator precedence is star, concatenation, then union.

\subsection{Examples}
\textbf{Example 1.} $\Sigma$ describes the language consisting of all strings with length 1 over alphabet $\Sigma$.\medskip
\\\textbf{Example 2.} $\Sigma^*$ describes the language consisting of all strings over $\Sigma$.\medskip
\\\textbf{Example 3.} $\Sigma^*1$ is the language containing all strings that end in a 1.\medskip
\\\textbf{Example 4.} $(0\Sigma^*) \cup (\Sigma^*1)$ consists of all strings that start with a 0 or end with a 1.\medskip
\\\textbf{Example 5.} $(\Sigma\Sigma\Sigma)^* = \{w | \text{the length of } w \text{ is a multiple of 3}\}$\medskip
\\\textbf{Example 6.} $1^*(01^+)^* = \{w |\text{every 0 in } w \text{ is followed by at least one 1}\}$\medskip
\\\textbf{Example 7.} $1^*\emptyset = \emptyset$. Concatenating the empty set to any set yields the empty set.\medskip
\\\textbf{Example 8.} $\emptyset^* = \{\epsilon\}$. The star operation puts together any number of strings from the language to get a string in the result. As the language is empty, the star operation can put together 0 strings, giving only the empty string.\medskip
\\\textbf{Example 9.} $R \cup \emptyset = R$. Adding the empty language to any language will not change it.\medskip
\\\textbf{Example 10.} $R \circ \epsilon = R$. Joining the empty string to any string will not change it.\medskip
\\\textbf{Example 11.} $R \cup \epsilon$ may not equal $R$. For example, if $R = 0$, $L(R) = \{0\}$, but $L(R \cup \epsilon) = \{0, \epsilon\}$.\medskip
\\\textbf{Example 12.} $R \circ \emptyset$ may not equal $R$. For example, if $R = 0$, $L(R) = \{0\}$, but $L(R \circ \emptyset) = \emptyset$.\medskip
\\\textbf{Example 13.} $(0 \cup \epsilon)1^* = 01^* \cup 1^*$. Expression $0 \cup \epsilon$ describes language $\{0, \epsilon\}$, so the concatenation operation adds either $0$ or $\epsilon$ before every string in $1^*$.\medskip
\\\textbf{Example 13.} $\Sigma^*001\Sigma^* = \{w | w \text{ contains the string } 001 \text{ as a substring}\}$.\medskip
\\\textbf{Example 14.} $01 \cup 10 = \{01, 10\}$.\medskip
\\\textbf{Example 15.} $(\Sigma\Sigma)^* = \{w|w \text{ is a string of even length}\}$. The length of a string is the number of symbols it contains.

\subsection{Equivalence with finite automata}
Regular expressions and finite automata have equivalent descriptive power. Any regular expression can be converted into a finite automaton that recognises the language it describes, and vice versa.\medskip
\\ A regular language is a language recognised by some finite automaton.\medskip
\\\textbf{Theorem.} A language is regular if and only if some regular expression describes it.\medskip
\\\textbf{Proof.}
\allowdisplaybreaks
\begin{align*}
(\rightarrow)\qquad&\parbox[t]{0.9\textwidth}{Suppose a regular expression $R$ describing some language $A$. Convert $R$ into an NDFA recognising $A$; if an NDFA recognises $A$, it is a regular language.
	\\ There are six cases in the formal definition of regular expressions:
	\begin{enumerate}
	\item $R = a$ for some $a \in \Sigma$. $L(R) = \{a\}$.
	\\ NDFA $N = (\{q_1, q_2\}, \Sigma, \delta, q_1, \{q_2\})$, where $\delta(q_1, a) = \{q_2\}$ and $\delta(r, b) = \emptyset$ for $r \neq q_1$ or $b \neq a$, recognises $L(R)$.
	\item $R = \epsilon$. $L(R) = \{\epsilon\}$.
	\\ NDFA $N = (\{q_1\}, \Sigma, \delta, q_1, \{q_1\})$ where $\delta(r, b) = \emptyset$ for any $r$ and $b$, recognises $L(R)$.
	\item $R = \emptyset$. $L(R) = \emptyset$.
	\\ NDFA $N = (\{q\}, \Sigma, \delta, q, \emptyset)$, where $\delta(r, b) = \emptyset$ for any $r$ and $b$, recognises $L(R)$.
	\item $R = R_1 \cup R_2$.
	\\ The class of regular languages is closed under union.
	\\ Construct NDFA for $R$ from those for $R_1$ and $R_2$ and closure under union construction.
	\item $R = R_1 \circ R_2$.
	\\ The class of regular languages is closed under concatenation.
	\\ Construct NDFA for $R$ from those for $R_1$ and $R_2$ and closure under concatenation construction.
	\item $R = R_1^*$.
	\\ The class of regular languages is closed under star.
	\\ Construct NDFA for $R$ from $R_1$ and closure under star construction.
	\end{enumerate}
} \\
(\leftarrow)\qquad&\parbox[t]{0.9\textwidth}{Idea: because $A$ is regular, it is accepted by a DFA. Describe a procedure for converting DFAs intk equivalent regular expressions.\medskip
\\ This part of the proof is given in the next section on generalised nondeterministic finite automata.}
\end{align*}

\section{Generalised nondeterministic finite automata}
\textbf{Generalised nondeterministic finite automata} (GNFAs) are nondeterministic finite automata where transition arrows may have any regular expressions as labels, instead of just members of the alphabet or $\epsilon$.\medskip
\\ GNFAs read blocks of symbols from the input, not necessarily just one symbol at a time. A GNFA moves along a transition arrow connecting two states by reading a block of symbols from the input, which themselves constitute a string described by the regular expression on that arrow.\medskip
\\ GNFAs are nondeterministic so may have several different ways to process the same input string.\medskip
\\ A GNFA accepts its input if its processing can cause the GNFA to be in an accept state at the end of the input.\medskip
\\Formally, a \textbf{generalised nondeterministic finite automaton} is a 5-tuple $(Q, \Sigma, \delta, q_{start}, q_{accept})$ where:
\begin{enumerate}
	\item $Q$ is the finite set of states
	\item $\Sigma$ is the input alphabet
	\item $\delta: (Q - \{q_{accept}\}) \times (Q - \{q_{start}\}) \rightarrow R$ is the transition function
	\item $q_{start}$ is the start state
	\item $q_{accept}$ is the accept state
\end{enumerate}
$R$ is the collection of all regular expressions over alphabet $\Sigma$. If $\delta(q_i, q_j) = R$, the arrow from state $q_i$ to state $q_j$ has regular expression $R$ as its label.\medskip
\\ The domain of the transition function is $(Q - \{q_{accept}\}) \times (Q - \{q_{start}\})$ as an arrow connects every state to every other state, except that no arrows come from $q_{accept}$ or go to $q_{start}$.\medskip
\\A GNFA \textbf{accepts} a string $w$ in $\Sigma^*$ if $w = w_1w_2...w_k$, where each $w_i$ is in $\Sigma^*$ and a sequence of states $q_0, q_1, ..., q_k$ exists such that:
\begin{enumerate}
	\item $q_0 = q_{start}$ is the start state
	\item $q_k = q_{accept}$ is the accept state
	\item for each $i$, $w_i \in L(R_i)$, where $R_i = \delta(q_{i-1}, q_i)$; $R_i$ is the expression on the arrow from $q_{i-1}$ to $q_i$
\end{enumerate}

\subsection{Special form}
For convenience, require that GNFAs have a special form meeting the following conditions:
\begin{itemize}
	\item The start state has transition arrows going to every other state, but no arrows coming in from any other state
	\item There is only a single accept state, which has arrows coming in from every other state, but no arrows going to any other state.
	\item The accept state is not the same as the start state.
	\item Except for the start and accept states, one arrow goes from every state to every other state, and also from each state to itself.
\end{itemize}

\subsection{Converting DFAs to GNFAs}
A DFA can be converted to a GNFA in the special form:
\begin{itemize}
	\item Add a new start state with an $\epsilon$ arrow to the old start state
	\item Add a new accept state with $\epsilon$ arrows from the old accept states
	\item If any arrows have multiple labels, or if there are multiple arrows going between the same two states in the same direction, replace each with a single arrow whose label is the union of the previous labels
	\item Add arrows labelled $\emptyset$ between states that had no arrows. This doesn't change the language recognised as a transition labelled with $\emptyset$ can never be used.
\end{itemize}

\subsection{Converting GNFAs to regular expressions}
Suppose a GNFA with $k$ states. As a GNFA must have a start and an accept state, which must be different from each other, $k \geq 2$.
\begin{itemize}
	\item If $k > 2$, construct an equivalent GNFA with $k - 1$ states. This step can be repeated on the new GNFA until it is reduced to two states.
	\item If $k = 2$, the GNFA has a single arrow going from the start state to the accept state -- the label of this arrow is the equivalent regular expression.
\end{itemize}
Cruicially, constructing an equivalent GNFA with one less state (when $k > 2$) is done by selecting a state, removing it from the machine, and repairing the remainder of the machine so that the same language is still recognised.\medskip
\\ Call the removed state $q_{rip}$. After removing this state, alter the regular expressions that label the remaining arrows to repair the machine. These labels should compensate the absence of the state and add back the lost computations.\medskip
\\ The new label, going from state $q_i$ to state $q_j$, is a regular expression that would take the machine from $q_i$ to $q_j$, either directly or through $q_{rip}$.\medskip
\\\texttt{CONVERT(G):}
\begin{enumerate}
	\item Let $k$ be the number of states in $G$
	\item If $k=2$, $G$ must consist of a start state, an accept state, and a single arrow connecting them and labelled with a regular expression $R$. Return the expression $R$.
	\item If $k > 2$, select any state $q_{rip} \in Q$ different from $q_{start}$ and $q_{accept}$. Let $G'$ be the GNFA $(Q', \Sigma, \delta', q_{start}, q_{accept})$, where $Q' = Q - \{q_{rip}\}$ for any $q_i \in Q' - \{q_{accept}\}$.
	\\ For any $q_j \in Q' - \{q_{start}\}$, let $\delta'(q_i, q_j) = (R_1)(R_2)^*(R_3)\cup(R_4)$, for $R_1 = \delta(q_i, q_{rip})$, $R_2 = \delta(q_{rip}, q_{rip})$, $R_3 = \delta(q_{rip}, q_j)$, and $R_4 = \delta(q_i, q_j)$.
	\item Compute \texttt{CONVERT(G')} and return this value.
\end{enumerate}
Now, prove that \texttt{CONVERT} returns a correct value.\medskip
\\\textbf{Claim.} For any GNFA $G$, \texttt{CONVERT(G)} is equivalent to $G$.\medskip
\\\textbf{Proof.} Prove this by induction on $k$, the number of states of the GNFA.\medskip
\\\textit{\textbf{Basis:}} If $k=2$, $G$ can only have a single arrow, which goes from the start state to the accept state. The regular expression label on this arrow describes all strings that allow $G$ to get to the accept state -- thus, this expression is equivalent to $G$.\medskip
\\\textit{\textbf{Induction step:}} Assume the claim is true for $k-1$ states -- use this assumption to prove the claim is true for $k$ states. First show that $G$ and $G'$ recognise the same language.\medskip
\\Suppose $G$ accepts an input $w$. Then, in an accepting branch of the computation, $G$ enters a sequence of states $q_{start}, q_1, q_2, ..., q_{accept}$.
\begin{itemize}
	\item If none of them are the removed state $q_{rip}$, $G'$ also accepts $w$ as each of the new regular expressions labelling the arrows of $G'$ contains the old regular expression as part of a union.
	\item If $q_{rip}$ appears, removing each run of consecutive $q_{rip}$ states forms an accepting computation for $G'$. The states $q_i$ and $q_j$ bracketing a run have a new regular expression on the arrow between them -- this describes all strings taking $q_i$ to $q_j$ via $q_{rip}$ on $G$. Thus, $G'$ accepts $w$.
\end{itemize}
Suppose $G'$ accepts an input $w$. As each arrow between any two states $q_i$ and $q_j$ in $G'$ describes the collection of strings taking $q_i$ to $q_j$ in $G$, either directly or through $q_{rip}$, $G$ must also accept $w$. Therefore, $G$ and $G'$ are equivalent.\medskip
\\The induction hypothesis states that when the algorithm recursively calls itself on input $G'$, the result is a rgular expression equivalent to $G'$, as $G'$ has $k-1$ states. Thus, this regular expression is equivalent to $G$, and the algorithm is correct.

\section{Nonregular languages}
Finite automata are limited in power -- can prove that certain languages cannot be recognised by finite automata.\medskip
\\ Consider language $B = \{0^n1^n | n \geq 0\}$. A DFA recognising $B$ would need to remember how many $0$s (not limited) have been seen so far as it reads the input; there are an unlimited number of possibilities to track. This can't be done with finitely many states.

\subsection{Pumping lemma}
The pumping lemma states that all regular languages have a special property. The property states that all strings in the language can be \textit{pumped} if they are at least as long as the \textit{pumping length}, a certain special value.\medskip
\\That means that each such string contains a section that can be repeated any number of times, with the resulting string remaining in the language.\medskip
\\If a language does not have this property, it is guaranteed not to be regular.\medskip
\\\textbf{Pumping lemma.} If $A$ is a regular language, there is a number $p$ (the pumping length), where if $s$ is any string in $A$ of at least length $p$, then $s$ may be divided into three pieces $s = xyz$, satisfying:
\begin{enumerate}
	\item for each $i \geq 0$, $xy^iz \in A$,
	\item $|y| > 0$, and
	\item $|xy| \leq p$.
\end{enumerate}
Note that $|s|$ represents the length of string $s$, $y^i$ means that $i$ copies of $y$ are concatenated together, and $y^0 = \epsilon$.\medskip
\\Either $x$ or $z$ may be $\epsilon$, but condition 2 says that $y \neq \epsilon$. Without this condition, the thorem would be trivially true.\medskip
\\Condition 3 states that the pieces $x$ and $y$ togrther have length at most $p$. This is an extra technical condition that is occasionally useful when proving certain languages to be nonregular.

\end{document}