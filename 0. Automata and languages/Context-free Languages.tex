\documentclass{article}
\usepackage{amsmath}
\usepackage{amssymb}
\begin{document}
\section{Introduction}
\noindent A \textbf{grammar} consists of a collection of \textbf{subsitution rules}, also referred to as productions. Each rule appears as a line in the grammar, comprising a \textbf{variable} (a symbol) and a string separated by an arrow. The string consists of variables and other symbols called \textbf{terminals}.\medskip
\\ The variable symbols are often represented by capital letters, and one variable is designated as the \textbf{start variable}, which usually occurs on the left-hand side of the topmost rule. The terminals are often represented by lowercase letters, numbers, or special symbols, and are analogous to the input alphabet.\medskip
\\ All strings generated by a grammar constitute the language of the grammar -- write $L(G_1)$ for the language of grammar $G_1$. A grammar is used to describe a language by generating each string of that language in the following manner:
\begin{itemize}
	\item Write down the start variable
	\item Find a variable that is written down and a rule that starts with that variable. Repalce the written down variable with the right-hand side of that rule
	\item Repeat step 2 until no variables remain
\end{itemize}
The sequence of substitutions to obtain a string is called a \textbf{derivation}. This information can be represented pictorially with a \textbf{parse tree}.

\section{Context-free grammars}
A \textbf{context-free grammar} is a 4-tuple $(V, \Sigma, R, S)$ where:
\begin{itemize}
	\item $V$ is a finite set called the variables,
	\item $\Sigma$ is a finite set, disjoint from $V$, called the terminals,
	\item $R$ is a finite set of rules, with each rule being  variable and a string of variables and terminals, and
	\item $S \in V$ is the start variable.
\end{itemize}
Any language that can be generated by some context-free grammar is a \textbf{context-free language}.
\end{document}