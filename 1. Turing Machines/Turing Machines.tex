\documentclass{article}
\usepackage{amsmath}
\begin{document}
\section{Formal definition of a Turing machine}
A Turing machine is a 7-tuple ($Q$, $\Sigma$, $\Gamma$, $\delta$, $q_0$, $q_{accept}$, $q_{reject}$):
\begin{itemize}
	\item $Q$ is the set of states
	\item $\Sigma$ is the input alphabet \textit{not} containing special blank symbol $\sqcup$
	\item $\Gamma$ is the tape alphabet satisfying $\Sigma \subset \Gamma$ and $\sqcup \in \Gamma$
	\item $\delta : Q \times \Gamma \rightarrow Q \times \Gamma \times \{L, R\}$ is the transition function
	\item $q_0 \in Q$ is the start state
	\item $q_{accept} \in Q$ is the accept state
	\item $q_{reject} \in Q$ is the reject state, $q_{reject} \neq q_{accept}$
\end{itemize}
The input alphabet $\Sigma$ never contains $\sqcup$, so $\Sigma \neq \Gamma$ is always true.
\\ A Turing machine can never contain a single state, as any machine must have distinct states $q_{accept}$ and $q_{reject}$.
\section{Turing machine computation}
\textbf{Tape content} is unbounded but always finite, and the first (leftmost) blank symbol marks the end of tape content.\medskip
\\A configuration $C_1$ yields the configuration $C_2$ if the Turing machine can legally go from $C_1$ to $C_2$ in a single step. A \textbf{configuration} consists of:
\begin{itemize}
	\item The current state
	\item The tape content
	\item The head location
\end{itemize}
The head can be in the same location in two successive steps if the machine attempts to move its head off the left-hand end -- we assume, by definition, that it just stays in the same cell rather than throwing an error.\medskip
\\The \textbf{start configuration} on an input $w \in \Sigma^*$ consists of start state $q_0$, $w$ as the tape content, and the head location is the first (leftmost) position of the tape.\medskip
\\A configuration is \textbf{accepting} if its state is $q_{accept}$. A configuration is \textbf{rejecting} if its state is $q_{reject}$.\medskip
\\Accepting and rejecting configurations are \textbf{halting} configurations.\medskip
%
\\ A Turing machine $M$ \textbf{accepts} an input $w$ if there is a sequence of configurations $C_1, C_2, ..., C_k$ such that:
\begin{enumerate}
	\item $C_1$ is the start configuration of $M$ on input $w$
	\item $C_i$ yields $C_{i+1}$ for $1 \leq i \leq k-1$
	\item $C_k$ is an accepting configuration
\end{enumerate}\medskip
The \textbf{language} of $M$, denoted $L(M)$, is the set of strings accepted by M.

\section{Turing-recognisable languages}
A language $L$ is \textbf{Turing-recognisable} if there is a Turing machine $M$ that recognises it, i.e. $L$ is the language of $M$.
\\Turing-recognisable means the same thing as \textbf{semi-decidable} and \textbf{recursively enumerable}.\medskip
\\ If $M$ recognises $L$, it may or may not halt on words not in $L$.

\subsection{Closure under operations}
\textbf{Example.} The collection of Turing-recognisable languages is closed under union.
\\\textbf{Proof.} Let $L_1$ and $L_2$ be Turing-recognisable languages and $M_1$ and $M_2$ be Turing machines that recognise them. Construct a Turing machine $M'$ that recognises the union of $L_1$ and $L_2$. On input $w$:
\begin{itemize}
	\item Run $M_1$ and $M_2$ alternatively on $w$, step-by-step. If either accept, accept. If both halt and reject, reject.
\end{itemize}
If either $M_1$ and $M_2$ accept $w$, $M'$ accepts $w$ and the accepting Turing machine (either $M_1$ or $M_2$) arrives to its accepting state after a finite number of steps.\medskip
\\ If both $M_1$ and $M_2$ reject and either does so by looping, $M'$ will loop.
\\ The solution for Turing-decidable languages would not work here as Turing machines can loop. If $M_1$ is looping, the construction used for Turing-decidable languages will loop even if $M_2$ accepts $w$, and thus, $w$ is the union of $L_1$ and $L_2$.

\section{Turing-decidable languages}
A language $L$ is \textbf{Turing-decidable} if there is a Turing machine $M$ that accepts every $w \in L$ and rejects every $w \notin L$.
\\ Turing-decidable means the same thing as \textbf{recursive}.\medskip
\\If $M$ decides $L$, it always halts.

\subsection{Closure under operations}
\textbf{Example.} The collection of decidable languages is closed under union.
\\\textbf{Proof.} For any two decidable languages $L_1$ and $L_2$, let $M_1$ and $M_2$ be the Turing machines that decide them. Construct a Turing machine $M'$ that decides the union of $L_1$ and $L_2$. On input $w$:
\begin{enumerate}
	\item Run $M_1$ on $w$. If it accepts, accept.
	\item Run $M_2$ on $w$. If it accepts, accept. Otherwise, reject.
\end{enumerate}
$M'$ accepts $w$ if either $M_1$ or $M_2$ accepts it. If both reject, then $M'$ rejects.

\section{Multitape Turing machines}
A \textbf{Multitape Turing machine} is like a single tape Turing machine with several tapes, each with its own head. The only difference in the formal definition is the transition function, which is now:
$$\delta: Q \times \Gamma^k \rightarrow Q \times \Gamma^k \times \{L, R\}^k$$
where k is the number of tapes.\medskip
\\\textbf{Theorem.} Every multitape Turing machine has an equivalent single tape Turing machine.

\section{Non-deterministic Turing machines}
A \textbf{non-deterministic Turing machine} has a transition function:
$$\delta: Q \times \Gamma \rightarrow \mathcal{P}(Q \times \Gamma \times \{L, R\})$$
\\\textbf{Theorem.} Every non-deterministic Turing machine has an equivalent deterministic Turing machine.
\\\textbf{Proof.} Consider the tree of all possible computations of the non-deterministic Turing machine. Start from the root (the start configuration) and do a breadth-first search. Accept only if an accepting configuration is found.
\begin{itemize}
	\item{DFS would not work}
	\item{Can use a multitape Turing machine to implement the BFS}
\end{itemize}

\section{Church-Turing thesis}
Intuitive notion of an algorithm is equivalent to the mathematical concept of an algorithm defined by Turing machines (or any other formal model of computation, such as $\lambda$-calculus, Post machines, recursive functions)

\section{Universal Turing machine}
Every Turing machine $M$ can be encoded as a word over a finite alphabet. Use $\langle M\rangle$ to denote the \textbf{encoding} of a Turing machine $M$.\medskip
\\\textbf{Theorem.} There is a Turing machine $U$ that takes a two-part input -- the encoding of a Turing machine $M$ ($\langle M \rangle$) and a word $w$, and simulate $M$ on $w$. $U$ is called a \textbf{universal Turing machine}.

\section{Halting problem}
\textbf{Halting problem:} Given an encoding of a Turing machine $M$ and a word $w$, does $M$ terminate on $w$?\medskip
\\\textbf{Proposition.} The Halting problem is Turing-recognisable.
\\\textbf{Proof.} Run a universal Turing machine on the pair ($\langle M\rangle, w$). Accept if the computation eventually terminates.\medskip
\\\textbf{Proposition.} The Halting problem is not Turing-decidable.
\\\textbf{Proof.} Assume, for contradiction, there is a turing machine $H$ that decides the Halting problem.
\begin{equation*}
H(\langle M \rangle, w) =
	\begin{cases}
		\text{accept} & \text{if } M \text{ terminates on } w \\
		\text{reject} & \text{if } M \text{ does not terminate on } w
	\end{cases}
\end{equation*}
Use $H$ as a black box to create an instance of the Halting problem, on which, $H$ fails.
\\Consider a Turing machine $D$ that takes the description of a single Turing machine $M$ as an input and does the following:
\begin{equation*}
D(\langle M\rangle) =
	\begin{cases}
		\text{accept} & \text{if } H(\langle M\rangle, \langle M\rangle) \text{ rejects}\\
		\text{loop} & \text{if } H(\langle M\rangle, \langle M\rangle) \text{ accepts}
	\end{cases}
\end{equation*}
What happens when D runs on its own encoding, $\langle D\rangle$?
\begin{enumerate}
	\item $D$ terminates on $\langle D\rangle$. By the construction of $D$, $H(\langle D\rangle, \langle D\rangle)$ rejects, giving a wrong answer
	\item $D$ does not terminate on $\langle D\rangle$. By the construction of $D$, $H(\langle D\rangle, \langle D\rangle)$ accepts, giving a wrong answer.
\end{enumerate}

\section{Turing-recognisable vs. Turing-decidable}
\textbf{Theorem.} A language $L$ is Turing-decidable if and only if both $L$ and its complement, $\bar{L}$, are Turing-recognisable.
\\\textbf{Proof.} Suppose $M_1$ recognises $L$, and $M_2$ recognises $\bar{L}$.
\\On an input $w$, run $M_1$ and $M_2$ in parallel -- i.e. simulate alternating steps of $M_1$ and $M_2$ on a multitape Turing machine.
\\Either $M_1$ or $M_2$ must eventually accept -- accept if $M_1$ accepts and reject if $M_2$ accepts.

\section{Co-Halting problem}
Given an encoding of a Turing machine $M$, and word $w$, is it the case that $M$ doesn't terminate on $w$, i.e. is not Turing-recognisable.

\section{Step-counter predicate}
Step ($M, w, k$) if and only if the machine $M$ terminates on $w$ in no more than $k$ steps, is Turing-decidable.

\end{document}