\documentclass{article}
\usepackage{amsmath}
\usepackage{amssymb}
\begin{document}
\section{m-reduciblity}
\textbf{Definition.} Let $A$ and $B$ be languages over alphabet $\Sigma$. $A$ is many-to-one reducible to $B$, written $A \leq B$, if there is a Turing machine $F$ that terminates on every input $u \in \Sigma^*$, and such that:
$$ A = \{u \in \Sigma^* | F(u) \in B\}$$
Informally, this means that checking $u \in A$ is no harder than checking $w \in B$.

\subsection{Properties}
\textbf{Proposition.} Suppose $A \leq B$.
\begin{enumerate}
	\item If $B$ is Turing-decidable, so is $A$
	\item If $B$ is Turing-recognisable, so is $A$
	\item If $A \leq B$ and $B \leq C$, then $A \leq C$
\end{enumerate}
Denote $A \equiv B$ to mean that $A \leq B$ and $B \leq A$. Informally, this means that $A$ and $B$ are equally difficult.

\section{m-completeness}
Language $A$ is \textbf{m-complete} if:
\begin{enumerate}
	\item $A$ is Turing-recognisable, and
	\item for every Turing-recognisable language $B$, $B \leq A$.
\end{enumerate}
Informally, if $A$ is m-complete, then $A$ is as hard as any other Turing-recognisable language\medskip
\\\textbf{Corollary.} If $A$ is m-comnplete and $A \leq B$, then $B$ is m-complete.\medskip
\\\textbf{Definition.} The Halting language $H$ consists of the words $\langle M\rangle \sqcup w$, over some fixed alphabet, such that the Turing machine $M$ terminates on $w$.\medskip
\\\textbf{Theorem.} H is m-complete
\\\textbf{Proof.} Generic reduction.
\\ Pick any Turing-recognisable language $A$. It is recognised by some machine $M_A$.
\\ Reduce it to $H$ by mapping any word $w$ to a word $\langle M_A\rangle \sqcup w$.
\\ The reduction is computable and $w \in A$ if and only if $\langle M_A\rangle \sqcup w \in H$.\medskip
\\\textbf{Definition.} $H_0$ is the diagonal of $H$, i.e. the language $\langle M\rangle \sqcup \langle M\rangle$ such that $M$ terminates on $\langle M\rangle$.\medskip
\\\textbf{Theorem.} $H_0$ is m-complete.
\\\textbf{Proof.} Reduction from $H$.
\\ Given a word $\langle M\rangle \sqcup w$, create a Turing machine $N_{M, w}$ that simulates $M$ on $w$. Note that this machine ignores the input. This can be done using a universal Turing machine.
\\ $N_{M, w}$ terminates on any input if and only if $M$ terminates on $w$.
\\ Thus, $N_{M, w}$ terminates on $\langle N_{M, w}\rangle$ if and only if $M$ terminates on $w$.

\section{Oracle Turing machine and t-reducibility}
\textbf{Definition.}
\begin{enumerate}
	\item An oracle for a language $A$ is a black box that takes word $w$ as input and instantly, and correctly, replies if $w \in A$
	\item An oracle Turing machine $M$, denoted $M^A$, is a Turing machine that has an additional capability of making calls to an oracle for the language $A$.
\end{enumerate}
\textbf{Definition.} A language $A$ is t-reducible to a language $B$ if $A$ is decidable by some oracle Turing machine $M^B$.\medskip
\\\textbf{Theorem.} If $A \leq_t B$, and $B$ is Turing-decidable, then $A$ is Turing-decidable

\section{Computable and partially computable functions}
\textbf{Definition.} A total function $f: \Sigma^* \rightarrow \Sigma^*$ is computable if there is a Turing machine $F$ such that on any input $x \in \Sigma^*$, $F$ produces $f(x)$ as output.\medskip
\\\textbf{Definition.} A partial function $g: \Sigma^* \rightarrow \Sigma^*$ is partially computable if there is a Turing machine $G$ such that on any input $x \in dom(g)$, $G$ produces $g(x)$ as the output and if $x \notin dom(g)$, $G$ doesn't terminate.\medskip
\\\textbf{Proposition.} A language (set) $S \subseteq \Sigma^*$ is Turing-recognisable if and only if it is:
\begin{itemize}
	\item The domain of a partially computable function
	\item The range of a computable function
	\item The range of a partially computable function
\end{itemize}

\section{Parameter theorem}
Let $M(x, y)$ be a Turing machine that expects the two-part input $x\sqcup y$. There is a Turing machine $SMN(t, x)$ that, on inputs $\langle M\rangle$ and $x$, produces a description of a Turing machine $\langle M_x\rangle$ such that for every $y$, $M_x(y) = M(x, y)$.
\\\textbf{Proof.} Follows from the recursion theorem.

\section{Recursion theorem}
Let $M(x, y)$ be a Turing machine that expects the two-part input $x\sqcup y$. There is a Turing machine $R(y)$ such that for every $y$, $R(y) = M(\langle R\rangle, y)$

\section{Partially computable functions without machines}
\textbf{Definition.} The \textbf{initial functions} are:
\begin{enumerate}
	\item The successor, $s(x) = x + 1$
	\item The zero, $n(x) = 0$
	\item The projections, $u_i^n(x_1, x_2, ..., x_n) = x_i$ for every $n \in \mathbb{N}$, $1 \leq i \leq n$.
\end{enumerate}

\section{Primitive recursive functions}
\textbf{Definition.} A function is called \textbf{primitive recursive} if it can be obtained from the initial functions by a finite number of applications of composition and primitive recursion.\medskip
\\ Let $f$ be a function on $k$ variables and $g_1, g_2, ..., g_k$ be functions on $n$ variables. The function $h$ on $n$ variables is obtained from $f$ and $g_1, g_2, ..., g_n$ by \textbf{composition} if:
$$h(x_1, x_2, ..., x_n) =^{def} f(g_1(x_1, x_2, ..., x_n), g_2(x_1, x_2, ..., x_n), ..., g_k(x_1, x_2, ..., x_n))$$
Let $f$ and $g$ be toptal functions on $n$ variables and $n+2$ variables respectively. The function $h$, on $n+1$ variables, is obtained from $f$ and $g$ by primitive recursion if:
$$h(x_1, x_2, ..., x_n, 0) =^{def} f(x_1, x_2, ..., x_n)$$
$$h(x_1, x_2, ..., x_n, t+1) = ^{def} g(t, h(x_1, x_2, ..., x_n, t), x_1, x_2, ..., x_n)$$
Primitive recursive functions are computable (find Turing machines to do this).

\section{Step-counter function is primitive recursive}
\textbf{Proposition.} The following functions are primitive recursive:
\begin{itemize}
	\item Addition
	\item Subtraction
	\item Multiplication
	\item Integral division (quotient and remainder)
	\item Exponentiation
	\item Integral logarithm
	\item n'th prime number
	\item i'th digit in base b expansion
\end{itemize}
\textbf{Definition.} The \textbf{Gödel number} of a sequence $x_1, ..., x_n$, defined as $p_1^{x_1}...p_{n-1}^{x_{n-1}}p_n^{x_{n+1}}$ where $p_i$ is the i'th prime number, is primitive recursive.\bigskip
\\ A string $w$ over a finite alphabet can be encoded by a single number $[w]$.
\\ A Turing machine $M$ can be encoded by a single number $[\langle M\rangle]$, shortened to $[M]$.\bigskip
\\ A configuration of a Turing machine $M$ -- consisting of a state, a head position, and tape content -- can be encoded as a single number $[q, i, w]$ via a primitive recursive function, $[q, i, w] = C(q, i, w)$.
\\ If a configuration $q, i, w$ yields a configuration $q', i', w'$, the function $Step([q, i, w]) = [q', i', w']$ is primitive recursive.\bigskip
\\ The step-counter function can be defined as:
$$SC([M], [w], 0) = [q_{start}, 0, w]$$
$$SC([M], [w], t+1) = Step(SC([M], [w], t))$$

\section{Gödel incompleteness}
In a formal system that reason about natural numbers, a formula $\phi$ is a finite sequence of symbols, so can be suitably encoded by a single number $[\phi]$.\bigskip
\\ A formula $\Pi$ in the system can be encoded by a single number $[\Pi]$. The predicate $\text{Proof}([\Pi], [\phi])$, stating that $\Pi$ is a proof of $\phi$, is primitive recursive.\bigskip
\\ Assume that the system is consistent (cannot prove both $\phi$ and $\neg\phi$ for any formula $\phi$. Assume that every \textit{true} formula system is provable, and assume that the system can reason about a Turing machine's computations.
\\ For every instance of the Halting problem, the predicate:
$$\exists p \text{ Proof}(p, [M \text{ does not terminate on } w])$$
is semi-decidable (recognisable).\smallskip
\\ For a positive instance, the predicate is true. By our assumption, there is a proof encoded by some number $p$. Find $p$ by brute force, trying 0, 1, ..., etc.\smallskip
\\ But then, the co-Halting problem is semi-decidable. However, as the Halting problem is semi-decidable, this would imply that the Halting problem is decidable -- a contradiction.

\section{Gödel's self-referential sentence}
Let $\phi(x)$ be a formula with one free variable $x$. Define the predicate $P(n, [\phi])$ to mean "$n$ doesn't encode a proof of $\phi([\phi])$."\medskip
\\ Define the formula $\psi(y)$ to be $\forall x P(x, y)$. It has a single free variable $y$, so consider $\psi([\psi])$.
\\ $\psi([\psi])$ says that every $x$ doesn't encode a proof of $\psi([\psi])$, i.e., $\psi([\psi])$ is not provable, and admits it!\medskip
\\ Is $\psi([\psi])$ false? No!
\\ If it were, what it says is false, thus it is provable. However, everything that is provable must be true -- a contradiction.
\\ Thus, $\psi([\psi])$ is true, so what it says is true, and $\psi([\psi])$ is not provable.

\section{Robinson arithmetic Q}
\textbf{Q} is the weakest sub-system of arithmetic in which Gödel's incompleteness holds.
\\ It has a constant zero 0 and function symbols $S$ for successor, $+$ for addition, and $\times$ for multiplication. It has no induction.

\subsection{Axioms}
\begin{enumerate}
	\item $S(x) \neq 0$
	\item $(S(x) = S(y)) \implies x=y$
	\item $y = 0 \lor \exists x (S(x) = y)$
	\item $x + 0 = x$
	\item $x + S(y) = S(x + y)$
	\item $x \times 0 = 0$
	\item $x \times S(y) = x \times y + x$
\end{enumerate}

\end{document}