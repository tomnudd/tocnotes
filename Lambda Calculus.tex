\documentclass{article}
\usepackage{amsmath}
\begin{document}
\section{Syntax of lambda calculus}
Assume a countable set of variable names, denoted by $a, b, c, ..., x, y, z, a_0, a_1, ...$\medskip
\\\textbf{Definition.} A \textbf{$\lambda$-term} is defined by the following context-free grammar:
\begin{align*}
<term>\qquad:&=\qquad <name>\\
 & |\qquad(\lambda <name> . <term>)\\
& |\qquad(<term><term>)
\end{align*}
\textbf{Conventions.}
\begin{enumerate}
	\item \textbf{Function application} is left-associative, so $(((A_1A_2)A_3)...A_k)$ can be abbreviated as $A_1A_2A_3...A_k$
	\item Nested \textbf{abstractions} $(\lambda x_1.(\lambda x_2.(...\lambda x_k. A)...))$ can be abbreviated as $\lambda x_1x_2...x_k.A$
\end{enumerate}
\textbf{Example.}
\\ $\lambda x y.F A B$ means $((\lambda x.(\lambda y.F)) A) B$

\section{Free variables}
\begin{enumerate}
	\item $<name>$ is free in $<name>$
	\item $<name>$ is free in $\lambda<name'>.<term>$ if $<name>\neq<name'>$ and $<name>$ is free in $<term>$
	\item $<name>$ is free in $<term'><term''>$ if $<name>$ is free in $<term'>$ or $<name>$ is free in $<term''>$
\end{enumerate}

\section{Bound variables}

\end{document}